% ----------------------------------------------------- 
%	加边框的命令
%	参考:https://tex.stackexchange.com/questions/531559/how-to-add-the-page-border-for-first-two-pages-in-latex
\usepackage{tikz}
\usetikzlibrary{calc}
\usepackage{eso-pic}
\AddToShipoutPictureBG{%
\begin{tikzpicture}[overlay,remember picture]
\draw[line width=0.6pt] % 边框粗细
    ($ (current page.north west) + (0.6cm,-0.6cm) $)
    rectangle
    ($ (current page.south east) + (-0.6cm,0.6cm) $); % 边框位置
\end{tikzpicture}}


\usepackage{xcolor}
\definecolor{c1}{HTML}{2752C9} % 目录颜色
\definecolor{c2}{RGB}{190,20,83} % 引用颜色

\usepackage{ctex}
\usepackage[top=28mm,bottom=28mm,left=15mm,right=15mm]{geometry} % 调整边距
\usepackage{hyperref} 
\hypersetup{
	colorlinks,
	linktoc = section, % 超链接位置,选项有section, page, all
	linkcolor = c1, % linkcolor 目录颜色
	citecolor = c1  % citecolor 引用颜色
}
\usepackage{amsmath,enumerate,multirow,float}
\usepackage{tabularx}
\usepackage{tabu}
\usepackage{subfig}
\usepackage{fancyhdr}
\usepackage{graphicx}
\usepackage{wrapfig}  
\usepackage{physics}
\usepackage{appendix}
\usepackage{amsfonts}
\usepackage{annotate-equations} % 公式标注
\usepackage{pgfplots}


% ---------------------------------------------------------------------
%	定义了两类colorbox
\usepackage{tcolorbox}
\tcbuselibrary{skins,breakable}
\newtcolorbox{tbox}[2][]{
    colframe=black!70!,
    breakable,
    enhanced,
	boxrule =0.5pt,
    title = {#2},
    fonttitle = \large\bfseries,
	drop fuzzy shadow,
    #1
}
\newtcolorbox[auto counter,number within=section]{question}[1][]{
  top=2pt,bottom=2pt,arc=1mm,
  boxrule=0.5pt,
%   frame hidden,
  breakable,
  enhanced, %跨页后不会显示下边框
  coltitle=c1!80!gray,
  colframe=c1,
  colback=c1!3!white,
  drop fuzzy shadow,
  title={思考题~\thetcbcounter:\quad},
  fonttitle=\bfseries,
  attach title to upper,
  #1
}

% ---------------------------------------------------------------------
%	利用cleveref改变引用格式,\cref是引用命令
\usepackage{cleveref}
\crefformat{figure}{#2{\textcolor{c2}{图 #1}}#3} % 图片的引用格式
\crefformat{equation}{#2{(\textcolor{c2}{#1})}#3} % 公式的引用格式
\crefformat{table}{#2{\textcolor{c2}{表 #1}}#3} % 表格的引用格式


% ---------------------------------------------------------------------
%	页眉页脚设置
\fancypagestyle{plain}{\pagestyle{fancy}}
\pagestyle{fancy}
\lhead{\kaishu 中山大学物理与天文学院物理实验\uppercase\expandafter{\romannumeral3}} % 左边页眉,学院 + 课程
\rhead{\kaishu Template 实验报告模板} % 右边页眉,实验报告标题
\cfoot{\thepage} % 页脚,中间添加页码
\setlength{\headheight}{13.6pt}

% ---------------------------------------------------------------------
%	对目录、章节标题的设置
\renewcommand{\contentsname}{\centerline{\huge 目录}}
\usepackage{titlesec}
\usepackage{titletoc}
% \titleformat{章节}[形状]{格式}{标题序号}{序号与标题间距}{标题前命令}[标题后命令]
\titleformat{\section}{\centering\LARGE}{}{1em}{}
\newcommand{\nsection}[3]{%
    \section{#1 #2 \hspace{11pt} \textbf{#3}}%
}

% ---------------------------------------------------------------------
%   listing代码环境设置
\usepackage{listings}
\lstloadlanguages{python}
\lstdefinestyle{pythonstyle}{
backgroundcolor=\color{gray!5},
language=python,
frameround=tftt,
frame=shadowbox, 
keepspaces=true,
breaklines,
columns=spaceflexible,                   
basicstyle=\ttfamily\small, % 基本文本设置,字体为teletype,大小为scriptsize
keywordstyle=[1]\color{c1}\bfseries, 
keywordstyle=[2]\color{Red!70!black},   
stringstyle=\color{Purple},       
showstringspaces=false,
commentstyle=\ttfamily\scriptsize\color{green!40!black},%注释文本设置,字体为sf,大小为smaller
tabsize=2,
morekeywords={as},
morekeywords=[2]{np, plt, sp},
numbers=left, % 代码行数
numberstyle=\it\tiny\color{gray}, % 代码行数的数字字体设置
stepnumber=1,
rulesepcolor=\color{gray!30!white}
}


% ---------------------------------------------------------------------
%	将表格封装起来
\newcommand{\scoresTable}[8]{
    \begin{table}
        \renewcommand\arraystretch{1.7}
        \begin{tabularx}{\textwidth}{
                |X|X|X|X
                |X|X|X|X|}
            \hline
            \multicolumn{2}{|c|}{预习报告} & \multicolumn{2}{|c|}{实验记录} & \multicolumn{2}{|c|}{分析讨论} & \multicolumn{2}{|c|}{总成绩} \\
            \hline
            \centering#1&\centering#2 &\centering#3 &\centering#4 &\centering#5 &\centering#6 &\centering#7 &{\centering#8} \\
            \hline
        \end{tabularx}
    \end{table}
}
\newcommand{\infoTable}[6]{
    \begin{table}
        \renewcommand\arraystretch{1.7}
        \begin{tabularx}{\textwidth}{|X|X|X|X|}
        \hline
        专业:& #1 &年级:& #2\\
        \hline
        姓名:& #3   & 学号:&#4\\
        \hline
        实验时间:&#5 & 教师签名:&#6 \\
        \hline
        \end{tabularx}
    \end{table}
}

% ---------------------------------------------------------------------
%	其他设置
\def\degree{${}^{\circ}$} % 角度
\graphicspath{{./images/}} % 插入图片的相对路径
\allowdisplaybreaks[4]  %允许公式跨页

\usetikzlibrary{patterns,decorations.markings,arrows.meta,bending}